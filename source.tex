\documentclass[10pt]{beamer}

\usepackage[croatian]{babel}
\usepackage[utf8]{inputenc}


\title{Git podržani protokoli i priprema servera}

\begin{document}
	\begin{frame}
		\maketitle	

	\end{frame}

	\begin{frame}{\centering Protokoli}
		\textbf{Git podržava četiri protokola}
		\begin{itemize}
			\item Lokalni protokol
			\item HTTP protokol
			\item SSH(secure shell) protol
			\item Git protokol
		\end{itemize}
	\end{frame}

	\begin{frame}{SSH protokol}
		\begin{itemize}
			

			\item Standardni protokol za prijenos podataka u slučaju da korisnik konfigurira Git server
			\item SSH pristup serverima je obično namješten, a ako nije, vrlo ga je jednostavan konfigurirati
			\item Način korištenja:    git clone ssh://[user@]server/project.git 
			\item Velik broj ljudi koji koriste Git su upoznati sa SSH protokolom 
			\item Siguran i kriptiran pristup i transfer podataka
			\item Nedostatak je što za pristup podatcima korisnika SSH protokola ostali korisnici moraju imati poseban SSH pristup  
		
		\end{itemize}		
	\end{frame}

	\begin{frame}[fragile]{Lokalni protokol}
		Koristi se kada je repozitorij podignut na računalu na koji svi članov tima imaju lokalni pristup.
		Članovi imaju pristup repozitoriju ako su ulogirani na istome računalu ili ako koriste disk koji je \textit{share-an} svim članovima tima.
		\newline \newline
		Naredba za kloniranje lokalnog repozitorija: \verb|$ git clone /srv/git/project.git|
		
	\end{frame}

	\begin{frame}{Lokalni protokol}
		\begin{table}
			\caption{Prednosti i nedostaci}
			\begin{tabular}{||p{130pt}|p{130pt}||}
				\hline
				\textbf{Prednosti} & \textbf{Nedostaci} \\ \hline
				Jednostavno podizanje repozitorija. & Teško je pristupiti repozitoriju 	sa udaljenih lokacija. \\
				Lagano je dohvatiti repozitorij od kolege: & Konfiguriranje NFS-a(network file system) nije jednostavno\\ \texttt{ git pull /home/john/project}  & Svaki korisnik ima potpunu ovlast u mijenjanju git-ovih datoteka. \\
				\hline
			\end{tabular}
		\end{table}

	\end{frame}

\end{document}
